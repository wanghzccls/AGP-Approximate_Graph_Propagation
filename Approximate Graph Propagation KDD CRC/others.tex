\header{\bf Propagation on directed graph.}
Our generalized propagation structure can also extend to directed graph by $\pi=\sum_{i=0}^\infty w_i \cdot \left(\D^{-a}\tilde{\A}\D^{-b} \right)^i \cdot \vec{x},$
% \begin{equation}\label{eqn:pi_gen_directed}
% \vspace{-2mm}
% 	\begin{aligned}
% 		\pi=\sum_{i=0}^\infty w_i \cdot \left(\bm{D}^{-a}\cdot \tilde{\bm{A}} \cdot \bm{D}^{-b} \right)^i \cdot \vec{x}, 
% 	\end{aligned}
% \end{equation}
where $\bm{D}$ denotes the diagonal out-degree matrix, and $\tilde{\A}$ represents the adjacency matrix or its transition according to specific applications. For PageRank, single-source PPR, HKPR, Katz we set $\tilde{\A}=\A^\top$ with the following recursive equation: 
\vspace{-2mm}
\begin{equation}\nonumber
\vspace{-2mm}
	\begin{aligned}
		\vec{r}^{(i+1)}(v)=\sum_{u\in N_{in}(v)} \left(\frac{Y_{i+1}}{Y_i} \right) \cdot \frac{\vec{r}^{(i)}(u)}{d_{out}^{a}(v) \cdot d_{out}^b(u)}. 
	\end{aligned}
\end{equation}
where $N_{in}(v)$ denotes the in-neighbor set of node $v$ and $d_{out}(u)$ is the out-degree of node $u$. 
% In comparison, for the propagation structure starting from a given target node, such as single-target PPR query,  $\tilde{\bm{A}}$ is set as $\bm{A}$ and 
% \begin{equation}\nonumber
% \vspace{-2mm}
% 	\begin{aligned}
% 		\vec{r}^{(i+1)}(v)=\sum_{u\in N_{out}(v)} \left(\frac{Y_{i+1}}{Y_i} \right) \cdot \frac{\vec{r}^{(i)}(u)}{d_{out}^{a}(v) \cdot d_{out}^b(u)}, 
% 	\end{aligned}
% \end{equation}
% where in each iteration, we update the residues of in-neighbors. 
% Beyond that, the propagation process and the complexity analysis are the same with that on undirected graphs. 







%In detail, as for PageRank and single-source PPR query, the propagation equation is that
%\begin{equation}\nonumber
%	\begin{aligned}
%		\pi=\sum_{i=0}^\infty \alpha \left( 1-\alpha \right)^i \cdot \left(\mathbf{A}^\top \mathbf{D}^{-1} \right)^i \cdot \vec{x}, 
%	\end{aligned}
%\end{equation}
%where $\vec{x}=\left(\frac{1}{n}\cdot \vec{1}\right)$ for PageRank, and $\vec{x}=\vec{e}_s$ for single-source PPR query. 
%The propagation structure of HKPR is similar with single-source PPR, except $w_i=\frac{e^{-t}t^i}{i!}$ following the poisson distribution with constant $t$ as parameter. Katz set $\tilde{\mathbf{A}}=A^\top$

%For example, as for PageRank, single-source PPR query, Katz, HITS, HKPR and so on, $\tilde{\bm{A}}$ should be set as $\bm{A}^\top$ that each propagation update the residue and reserve of out-neighbors. 





%where $\tilde{N}(v)=N_{in}(v)$ if $\mathbf{\tilde{A}}=\mathbf{A}^\top$ in the generalized equation~\eqref{eqn:pi_gen}, or $\tilde{N}(v)=N_{out}(v)$ if $\mathbf{\tilde{A}}=\mathbf{A}$. 


%If the $\tilde{A}=A^\top$ in the propagation structure, such as the single-source PPR query, HITS, Katz, and HKPR, we add $\hat{R}^{(i+1)}(v)$ by $\left(1-\frac{w_i}{Y_i} \right) \cdot \frac{\hat{R}^{(i)}(u)}{d_v^{1-r} \cdot d^r_u}$ for every out-neighbors $v\in N_o(u)$ (Line 6-7). While if $\tilde{A}=A$ such as the single-target PPR query, we update the residue for every in-neighbor of node $u$. 


\begin{comment}
\begin{table*}[t]
	%\vspace{-2mm}
	\centering
	\renewcommand{\arraystretch}{1.5}
	\tblcapup
	\caption{Typical propagation equations.}
	\vspace{-3mm}
	\tblcapdown
%	\begin{small}
		\begin{tabular}{|c|c|c|c|c|c|c|c|} \hline
			\multicolumn{2}{|c|}{\bf Algorithm} & {\bf $\boldsymbol{\tilde{A}}$}& {\bf $\boldsymbol{a}$}& {\bf $\boldsymbol{b}$} & {\bf $\boldsymbol{w_i}$} & {\bf $\boldsymbol{\vec{x}}$} & {\bf Propagation equation} \\ \hline
			\multicolumn{2}{|c|}{PageRank} & $\mathbf{A}^\top$ & $a=0$ & $b=1$& $\alpha \left( 1-\alpha \right)^i $ & $\frac{1}{n}\cdot \vec{1}$ & $\pi=\sum_{i=0}^\infty \alpha \left( 1-\alpha \right)^i \cdot \left(\mathbf{A}^\top \cdot \mathbf{D}^{-1} \right)^i \cdot \left(\frac{1}{n}\cdot \vec{1} \right)$ \\ \hline
			\multicolumn{2}{|c|}{single-source PPR} &$\mathbf{A}^\top$ &$a=0$ & $b=1$ & $\alpha \left( 1-\alpha \right)^i $ & one-hot vector $\vec{e}_s$ & $\pi=\sum_{i=0}^\infty \alpha \left( 1-\alpha \right)^i \cdot \left(\mathbf{A}^\top  \cdot \mathbf{D}^{-1} \right)^i \cdot \vec{e}_s $ \\ \hline
			\multicolumn{2}{|c|}{single-target PPR} &$\mathbf{A}$ & $a=1$ & $b=0$ & $\alpha \left( 1-\alpha \right)^i $ & one-hot vector $\vec{e}_t$ & $\pi=\sum_{i=0}^\infty \alpha \left( 1-\alpha \right)^i \cdot \left(\mathbf{D}^{-1} \cdot \mathbf{A} \right)^i \cdot \vec{e}_t $\\ \hline
			\multicolumn{2}{|c|}{HKPR} &$\mathbf{A}^\top$ &$a=0$ & $b=1$ & $e^{-t} \cdot \frac{t^i}{i!} $ & one-hot vector $\vec{e}_s$ & $\pi=\sum_{i=0}^\infty e^{-t} \cdot \frac{t^i}{i!}\cdot \left(\mathbf{A}^\top  \cdot \mathbf{D}^{-1} \right)^i \cdot \vec{e}_s $\\ \hline
			%\multicolumn{2}{|c|}{$k$-hop transition probability} & $r=1$ & $w_i=0 (i \neq k), w_k=1$ & one-hot vector $\vec{e}_s$ & $\left(\mathbf{A} \cdot \mathbf{D}^{-1} \right)^k \cdot \vec{e}_s $\\ \hline
			\multirow{3}{*}{GNN} & SGC,GCN &$\mathbf{A}^\top$ & $a=\frac{1}{2}$ & $b=\frac{1}{2}$ & $w_i=0 (i \neq k), w_k=1$ & feature vector $\vec{x}$ & $\pi=\left(\mathbf{D}^{-\frac{1}{2}} \cdot \mathbf{A} \cdot \mathbf{D}^{-\frac{1}{2}} \right)^k \cdot \vec{x} $\\ \cline{2-8}
			~& APPNP &$\mathbf{A}^\top$ &  $a=\frac{1}{2}$ & $b=\frac{1}{2}$ & $\alpha \left( 1-\alpha \right)^i$ & feature vector $\vec{x}$ & $\pi=\sum_{i=0}^k \alpha \left( 1-\alpha \right)^i \cdot \left(\mathbf{D}^{-\frac{1}{2}} \cdot \mathbf{A} \cdot \mathbf{D}^{-\frac{1}{2}} \right)^i \cdot \vec{x} $ \\ \cline{2-8}
			~& GDC &$\mathbf{A}^\top$ &  $a=\frac{1}{2}$ & $b=\frac{1}{2}$ & $e^{-t} \cdot \frac{t^i}{i!} $  & feature vector $\vec{x}$ & $\pi=\sum_{i=0}^k e^{-t} \cdot \frac{t^i}{i!}\cdot \left(\mathbf{D}^{-\frac{1}{2}} \cdot \mathbf{A} \cdot \mathbf{D}^{-\frac{1}{2}} \right)^i \cdot \vec{x} $ \\ \hline
			\multicolumn{2}{|c|}{Katz} &$\mathbf{A}^\top$ & $a=0$ & $b=0$ & $\beta^i$ & one-hot vector $\vec{e}_s$ & $\pi=\sum_{i=0}^\infty \beta^i \cdot \left(\mathbf{A}^\top \right)^i \cdot \vec{x}$\\ \hline
			\multicolumn{2}{|c|}{HITS} &$\mathbf{A}^\top$ & $a=0$ & $b=0$ & $w_i=\frac{1}{\mu_{j}}(i=2j)$, or $w_i=0$ & $\frac{1}{n}\cdot \vec{1}$ & $\pi=\sum_{j=0}^\infty \frac{1}{\mu_j}\cdot \left(\mathbf{A}^\top\right)^{2j} \cdot \vec{1}$ \\ \hline	
		\end{tabular}
%	\end{small}
	%\label{tbl:propagation}
	%\tbldown
	%\vspace{-3mm}
\end{table*}

\end{comment}



%%% Local Variables:
%%% mode: latex
%%% TeX-master: "paper"
%%% End:
